\chapter{جمع‌بندی و کار‌های آتی}\label{Chap:Chap6}
\minitoc

در این پروژه سعی شد تا ضمن بررسی مدل‌های زبانی به صورت شرطی، با رویکردی جدید به حل مسئله پرداخته شود. در ابتدا با مقایسه رویکرد‌های مختلف از جمله مدل‌های با فضای نهان و یا فاقد آن، آموزش بر مبنای \maxlikelihood{} و یا \gan{}، مشکلات و سختی‌های هر کدام شرح داده شد. \gan{} با سختی‌های آموزش از جمله \modecollapse{} و میزان آموزش \discriminator{} و \generator{} رنج برده که به عنوان راه حل، \wgan{} معرفی شد. با این وجود این راه حل به دلیل ذات گسسته متن و عدم انتقال گرادیان از \discriminator{} به \generator{} همچنان در حوزه متن با مشکل مواجه است.
در مدل‌های با فضای نهان، مشکل عدم توجه به فضای نهان وجود داشت که برای این مشکل نیز راه حل‌های متفاوتی ارائه شد. در نهایت به منظور داشتن کنترل بر روی خروجی مدل، مدل‌های همراه با فضای نهان انتخاب شده و همچنین برای دوری از مشکل عدم توجه به فضای نهان نیز روش \wae{} برگزیده شد که به لحاظ نظری پشتوانه قوی‌تری داشت. برای معماری مورد استفاده در \encoder{} و \decoder{} نیز حالت‌های مختلف مورد آزمایش قرار گرفت که در نهایت معماری \transformer{} در هر دو بخش \encoder{} و \decoder{} بهترین نتیجه را کسب کرد.
\\
برای داشتن یک مدل شرطی از دو دیدگاه به حل مسئله پرداخته شد. 
در دیدگاه اول، مقدار شرط فضای نهان را تعیین می‌کند. به این منظور اقدام به آموزش یک مولد شرطی فضای نهان کرده که نسبت به مقادیر شرط تقسیم پذیر باشد؛ علاوه بر این برای یادگیری فضای نهان هر مقدار شرط و همچنین جلوگیری از رخداد \modecollapse{}، از خانواده دیگری از شبکه‌ها به نام \normalizingflownets{} و به طور دقیق‌تر از معماری \lr{MAF} که یکی از شبکه‌های رایج و قوی در این خانواده است و امکان آموزش بر اساس \maxlikelihood{} را دارند، معرفی و استفاده شد.
در دیدگاه دوم، مقدار شرط و فضای نهان از یکدیگر مستقلند. روش آموزش مانند آموزش \wae{} است؛ تنها با این تفاوت که \decoder{} شرطی بوده و علاوه بر بردار فضای نهان، بردار مقدار شرط را نیز دریافت می‌کند. به منظور استقلال فضای نهان از شرط نیز به روشی تخاصمی رفتار شده است. ضمن استفاده از یک \classifier{} در دسته‌بندی بردار‌های فضای نهان نسبت به مقدار شرط، \encoder{} در جهت مخالف سعی در تولید بردار‌های فضای نهانی که \classifier{} توانایی تشخیص برچسب آن‌ها را ندارد، داشته و عملا میزان حضور شرط در فضای نهان را تقلیل می‌دهد.
\\
با وجود اینکه مدل \sentigan{} که به ازای هر شرط یک مدل در نظر می‌گیرد و توسعه پذیر نیست، مدل پیشنهادی دوم ضمن داشتن تنها یک مدل مولد به ازای شروط مختلف، بالاترین درصد رعایت شرط را کسب کرده است. این در حالیست که کیفیت و تنوع جملات تقریبا در سطح سایر مدل‌ها حفظ شده است. در مورد مدل پیشنهادی اول نیز گرچه نتایج نزدیک به مدل پایه مبتنی بر فضای نهان کسب شده است، اما از روش کمتر مورد توجه \normalizingflownets{} جهت یادگیری فضای نهان بهره برده شده است.
\\
به عنوان کار‌های آتی، می‌تواند استفاده از مدل‌های قوی‌تر \normalizingflownets{} و یا ابداع و استفاده از روشی برای کم کردن فاصله توزیع پیشین با توزیع \marginal{} \encoder{} که عاری از مشکلاتی همچون \modecollapse{} ناشی از \gan{} بوده و همچنین نسبت به پارامتر‌های \encoder{} و \priordist{} قابل بهینه‌سازی است، مورد بررسی قرار گیرد. در نتیجه این موضوع، \encoder{} فضای نهان با محدودیت کمتری خواهد ساخت و به توانایی مدل‌سازی کمک شایانی می‌کند. گذشته از این موضوع می‌توان از راه‌کارهای حل \expbias{} برای رفع این موضوع نیز بهره برده و این مشکل مدل‌های پیشنهادی را نیز رفع کرد.