\thispagestyle{empty}
\centerline{\textbf{\large{چکیده}}}
\begin{quote}
در دهه اخیر، با ظهور روش­‌های یادگیری ژرف، در زمینه مدل‌­های زبانی پیشرفت شایانی صورت گرفته است که یکی از مهم‌ترین \task{}‌های پردازش زبان طبیعی محسوب می‌شود. به دلیل پایه‌ای‌تر بودن این \task{}، اخیرا شبکه‌های عظیمی بر مبنای این \task{}  \pretrain{} داده شده و برای \task{} مقصد \finetuning{} می‌شوند که این موضوع نشان از اهمیت و کاربرد آن حتی در سایر \task{}‌های این حوزه همچون، ترجمه، تحلیل تمایل، پرسش و پاسخ و غیره دارد.
 با این وجود به دلیل ذات گسسته فضای متن، مدل‌های زبانی با مشکلات متعددی مواجه هستند. روش‌های بر پایه \teacherforcing{} (\lr{Teacher Forcing}) با مشکل \expbias{} (\lr{Exposure Bias}) روبرو بوده که مربوط به تفاوت فرآیند آموزش و آزمون است. برای حل آن استفاده از رويکرد يادگيری تقويتی و \gan{} که آموزش سختی دارد و یا روش‌های تقریبی دیگری پيشنهاد شده است. در مدل‌های با فضای نهان نیز پدیده عدم توجه \decoder{} به فضای نهان گزارش شده است.
\\
\task{}
کاربردی‌­تر تولید شرطی جملات زبان طبیعی همچون شروط ساده­‌ای مانند زمان فعل جمله (حال و گذشته و آینده) تا شروط پیچیده‌­تری همچون زمینه و موضوع نیز از اهمیت ویژه‌ای برخوردار است. کاربرد این مدل‌ها تنها در حوزه متن نبوده و در زمینه تولید مولکول‌های دارویی، تولید موسیقی در یک ژانر خاص و یا تولید گراف با ویژگی‌های مشخص کمک کننده است. در مدل‌های شرطی نیز علاوه بر مشکلات ذکر شده برای مدل‌های زبانی غیر شرطی، امکان عدم تطابق جمله تولیدی با شرط را نیز باید افزود.
\\
در این پروژه سعی شده است برای \task{} تولید متن شرطی که شروط مقادیر گسسته‌ای دارند، مدلی \generative{} با فضای نهان، با دو دیدگاه مدل‌سازی متفاوت، آموزش داده شود. در روش اول، شرط، فضای نهان و فضای جملات به صورت سلسله‌مراتبی با یکدیگر رابطه دارند؛ به این صورت که شرط فضای نهان را تعیین کرده و فضای نهان جمله را تعیین می‌کند. به عبارت دیگر \priordist{} فضای نهان، نسبت به مقادیر مختلف شرط تقسیم شده است. در نهایت برای یادگیری  \priordist{} شروط مختلف از \normalizingflownets{} که اخیرا مورد توجه قرار گرفته‌اند، بهره برده شده است. اما در روش دوم فضای نهان و شرط از یکدیگر مستقل هستند و هر دو با هم یک جمله را تعیین می‌کنند. در این دیدگاه، محتوای جملات، غیر از مقدار شرط در فضای نهان مدل شده و این فضا نسبت به مقادیر شرط تقسیم نمی‌شود. از آنجا که در هر دو دیدگاه از مدل‌های \generative{} با فضای نهان استفاده شده است و در این مدل‌ها، امکان بروز مشکل عدم توجه به فضای نهان وجود دارد، از روشی به نام \wae{} استفاده شده است تا از بروز این مشکل جلوگیری شود. 
\\
در نهایت نیز مدل‌های پایه و مدل‌های پیشنهادی با استفاده از معیار‌های متفاوتی همچون کیفیت، تنوع جملات و درصد رعایت شرط مورد ارزیابی قرار گرفته‌اند که مدل پیشنهادی دوم ضمن رعایت کیفیت و تنوع جملات، در درصد رعایت شروط، از سایر مدل‌ها عملکرد بهتری داشته و مدل پیشنهادی اول نیز مشابه مدل پایه دارای فضای نهان عمل کرده و در بعضی دادگان در رعایت کیفیت و تنوع عملکرد بهتری دارد.
\vskip .5cm
\textbf{کلمات کلیدی:} \textiranic{
    تولید شرطی متن، مدل‌های مولد با فضای نهان، شبکه‌های عصبی، یادگیری ژرف}
\end{quote}