\thispagestyle{empty}
\centerline{\textbf{\large{چکیده}}}
\begin{quote}
در دهه اخیر، با ظهور روش­‌های یادگیری ژرف، در زمینه مدل‌­های زبانی پیشرفت شایانی صورت گرفته است: اما به دلیل ذات گسسته فضای متن، مدل‌های زبانی با مشکلات متعددی مواجه هستند. روش‌های بر پایه \teacherforcing{} با مشکل \expbias{} روبرو بوده که مربوط به تفاوت فرآیند آموزش و آزمون است. برای حل آن استفاده از رويکرد يادگيری تقويتی و \gan{} که آموزش سختی دارد و یا راه‌حل‌های تقریبی دیگری پيشنهاد شده است. در مدل‌های با فضای نهان نیز پدیده عدم توجه \decoder{} به فضای نهان به وجود می‌آید.
\\
اگر بخواهیم یک گام جلو رفته و هدف پیشرفته­‌تر و کاربردی‌­تری در تولید جملات زبان طبیعی در نظر بگیریم، می‌­توان عملیات تولید متون را به صورت شرطی انجام داد؛ این شروط می­‌توانند از شروط ساده­‌ای مانند زمان فعل جمله (حال و گذشته و آینده) تا شروط پیچیده‌­تری همچون زمینه و موضوع را شامل شوند. در مدل‌های شرطی نیز علاوه بر مشکلات ذکر شده برای مدل‌های زبانی غیر شرطی، امکان عدم تطابق جمله تولیدی با شرط را نیز باید افزود.
\\
هدف از این پروژه ارائه مدلی \generative{} جهت تولید جملات شرطی است؛ اما این شرط برای مشخص نمودن محدوده پروژه به شروط گسسته محدود شده است. کاربرد این مدل‌ها تنها در حوزه متن نبوده و در زمینه تولید مولکول‌های دارویی، تولید موسیقی و یا تولید گراف با ویژگی‌های مشخص کمک کننده است.\\

\vskip 1cm
\textbf{کلمات کلیدی:} \textiranic{}
\end{quote}