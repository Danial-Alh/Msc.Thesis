\chapter{پیاده‌سازی، آزمایش‌ها و ارزیابی}\label{Chap:Chap4}
\minitoc
\section{مقدمه}
تا به حال به لحاظ نظری به نحوه حل مسئله پرداخته شد. در این فصل پس از معرفی دادگان آموزشی و معیارهای ارزیابی مورد استفاده، نحوه پیاده‌سازی و دشواری‌های پیش‌آمده هنگام آموزش شرح داده خواهند شد. در انتها نیز ضمن گزارش نتایج، عملکرد مدل‌ها با یکدیگر مقایسه شده‌اند.
\section{معیارهای ارزیابی}
\section{دادگان آموزشی}
\section{آموزش \wae{}}
همان طور که در فصل ؟؟؟ توضیح داده شد، آموزش شبکه از دو بخش کلی تشکیل شده است. اولین بخش مربوط به آموزش \wae{} است. به این منظور به عنوان اولین تلاش از معماری \lr{LSTM} با فضای نهان ؟؟؟ بعدی در \encoder{} و \decoder{} با بردارهای  \embedding{} ؟؟؟ تایی استفاده شد. مشکل‌هایی در آموزش این معماری پیش آمد که در ادامه توضیح داده خواهند شد. لازم به ذکر است همان طور که در فصل ؟؟؟ توضیح داده شد، \decoder{} بایستی 
$G(Z) = \argmax_{x} P_\psi(x|Z)$
و فضای خروجی $P_\psi$ یک بردار $|V|^{m}$ بعدی است که $m$ حداکثر طول جملات و $|V|$ نیز اندازه واژگان است؛ اما از آنجا که توانایی محاسباتی یافتن $\argmax$ بر روی چنین فضای بزرگی را نداریم بنابراین از روش 
\trans{\greedydecoding}{Greedy decoding}
که تقریبی از $\argmax$ بر روی کل فضاست استفاده می‌کنیم.
\subsection{\encoder{}،
عامل آموزش ناموفق}
پس از آموزش شبکه توضیح داده شده، معضل عدم توجه \decoder{} به فضای نهان پیش آمد. تصاویر ؟؟؟ مربوط به نتایج \bleu{} و \selfbleu{} در دو حالت گزارش شده است. در حالت اول یک مجموعه جمله از دادگان 
\trans{\validation{}}{Validation}
به فضای نهان برده شده و مجددا بازسازی ‌شده و \bleu{} و \selfbleu{} این مجموعه نسبت به داده آزمون اندازه‌گیری و گزارش شده است. این حالت را حالت بازسازی می‌نامیم. در حالت دوم تعدادی نمونه از \priordist{} گرفته شده و توسط \decoder{}  
\decode{}
شده و مجددا معیارهای ذکر شده گزارش شده‌اند. این حالت را نیز حالت نمونه‌برداری می‌نامیم.
مقادیر \bleu{} و \selfbleu{} دادگان \validation{} بر روی دادگان آزمون نیز به شرح زیر است:
عکس بلو سلف بلو :)
اعداد بلو سلف بلوی دیتا :)
\\
همان طور که در تصاویر ؟؟؟ مشخص است، با اینکه مقدار \bleu{} از دادگان اصلی بیشتر است (!) اما \selfbleu{} نزدیک به یک است! این موضوع نشان از این دارد که به ازای تغییرات $z$ در فضای نهان، خروجی \decoder{} تغییر چندانی نمی‌کند. مشاهده نمونه‌های تولید شده توسط مدل نیز گواهی بر این موضوع است.
گواه موضوع :)
\\
حدس اولیه بر این بود که اتفاقی شبیه به آنچه در \vae{} گزارش شده است، رخ داده است. به منظور آزمایش دقیق‌تر یک \autoencoder{} ساده و بدون هیچ تابع هزینه اضافی آموزش داده شد. نتایج به شرح زیر بدست آمد:
نتایج به شرح :)
\\
به وضوح مشخص است که پدیده مشابهی رخ داده است. بنابراین مشکل احتمالا از نحوه آموزش نیست. در گام بعدی تغییر معماری در \encoder{} و \decoder{} مورد بررسی قرار گرفت. معماری‌های مورد آزمایش \lr{LSTM}، \lr{CNN} و \lr{Transformer}  بودند. از آنجا که قدرت \lr{CNN} در تولید جمله در مقایسه با \lr{LSTM} کمتر است، از بررسی \lr{CNN} به عنوان \decoder{} صرف نظر شد. لازم به ذکر است که به دلیل اعمال نکردن توزیعی بر فضای نهان، اعداد برای حالت بازسازی گزارش شده‌اند.
عکس معماری‌های متفاوت و ترکیبشان :)
\\
نکته قابل توجه این است که گویا مشکل در معماری \encoder{} نهفته است. در صورت استفاده از معماری \transformer{} مشکل عدم توجه به فضای نهان تقریبا رفع شده و \bleu{} و \selfbleu{} در حالت بازسازی تقریبا برابر با مقادیر این معیار برای دادگان \validation{} است. نکته قابل توجه دیگر این است که معماری \lstm{} و \transformer{} در \decoder{} تفاوت چندانی ایجاد نمی‌کند و تنها مسیر حرکت متفاوتی دارند.
به دلیل همگرایی سریع‌تر معماری \transformer{}، این معماری برای سایر آزمایشات برگزیده شد. در ادامه با استفاده از معماری \transformer{} یک \wae{} آموزش داده شد که نتایج به شرح زیر است:
شرح آزمایش :)
\\
\subsection{استفاده از  \gan{} به جای \mmd{}}
در بخش‌های متفاوتی صحبت از رفتار \modecollapse{} 
\gan{}
سخن به میان آمد. در اینجا طی یک آزمایش از \lr{WGAN} به جای \mmd{} برای یادگیری فضای نهان یک \autoencoder{} استفاده شد. به عبارت دیگر اگر $F(\epsilon)$ شبکه‌ای باشد که با گرفتن یک نوفه با توزیع گاوسی نرمال، آن را به نمونه‌ای در فضای نهان تبدیل کند و $D(Z)$ یک \critic{} بین نمونه‌های تولید شده توسط $F$ و فضای نهان ساخته شده توسط \encoder{} 
$Q(Z|X)$
باشد، تابع هزینه ذیل استفاده گشت:
\begin{gather}
 \mathcal{L}_\text{WGAN} (F, D)= 
   \expected_{x \sim P_\text{Data}(X), z \sim Q(Z|x)} D(z)
 - \expected_{\epsilon \sim N(\bff{0}, \bff{I}), z \sim F(\epsilon)} D(z)
  \\
\text{\lr{s.t: D is 1-Lipschitz}} \nonumber
\end{gather}
که نسبت به $D$ بیشینه و نسبت به $F$ کمینه می‌گردد. برای برآوردن شرط \lr{1-Lipschitz} بودن $D$ نیز از روش \lr{Gradient Penalty} استفاده شده است. اگر $\bff{z}$ و $\tilde{\bff{z}}$ به ترتیب نمونه‌هایی از $Q$ و $F$ ،
 $t$
  متغیر تصادفی از توزیع یکنواخت $[0,1]$ و 
  $\hat{z} = t\bff{z} + (1 - t) \tilde{\bff{z}}$
  باشد، \lr{Gradient Penalty} به صورت زیر نوشته می‌شود:
\begin{gather}
\mathcal{L}_\text{WGAN-GP} (D; \bff{z}, \tilde{\bff{z}}) = \lambda_{gp} \expected_{t \sim U(0, 1)}
(|| \nabla_{\hat{\bff{z}}} D(\hat{\bff{z}}) ||_2  - 1)^2
\end{gather}
که در کنار تابع هزینه اصلی بهینه‌سازی می‌شود. نتایج آزمایش فوق به شرح زیر است:
شرح آزمایش :)
\\
\subsection{اثر تابع هزینه  \lr{ExpBias}}
\section{آموزش مولد شرطی}
\subsection{تکنیک‌های آموزش شبکه}

\section{نتایج و مقایسه با سایر مدل‌ها}