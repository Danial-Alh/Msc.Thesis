\chapter{جمع‌بندی و کار‌های آتی}\label{Chap:Chap6}
\minitoc

در این پروژه سعی شد تا ضمن بررسی مدل‌های زبانی به صورت شرطی، با رویکردی جدید به حل مسئله پرداخته شود. در ابتدا با مقایسه رویکرد‌های مختلف از جمله مدل‌های با فضای نهان و یا فاقد آن، آموزش بر مبنای \maxlikelihood{} و یا \gan{}، مشکلات و سختی‌های هر کدام شرح داده شد. \gan{} با سختی‌های آموزش از جمله \modecollapse{} و میزان آموزش \discriminator{} و \generator{} رنج برده که به عنوان راه حل، \wgan{} معرفی شد. با این وجود این راه حل به دلیل ذات گسسته متن و عدم انتقال گرادیان از \discriminator{} به \generator{} همچنان در حوزه متن با مشکل مواجه است.
در مدل‌های با فضای نهان، مشکل عدم توجه به فضای نهان وجود داشت که برای این مشکل نیز راه حل‌های متفاوتی ارائه شد. در نهایت به منظور داشتن کنترل بر روی خروجی مدل، مدل‌های همراه با فضای نهان انتخاب شده و همچنین برای دوری از مشکل عدم توجه به فضای نهان نیز روش \wae{} برگزیده شد که به لحاظ نظری پشتوانه قوی‌تری داشت. برای معماری مورد استفاده در \encoder{} و \decoder{} نیز حالت‌های مختلف مورد آزمایش قرار گرفت که در نهایت معماری \transformer{} در هر دو بخش \encoder{} و \decoder{} بهترین نتیجه را کسب کرد.
\\
برای داشتن یک مدل شرطی اقدام به آموزش یک مولد شرطی فضای نهان کرده و برای جلوگیری از رخداد \modecollapse{} از خانواده دیگری از شبکه‌ها به نام \normalizingflownets{} که امکان آموزش بر اساس \maxlikelihood{} را دارند، معرفی و استفاده شد. به این منظور برای راحت‌تر کردن توزیع شرطی در فضای نهان، سعی در تقسیم فضای نهان با توجه به مقادیر شرط شد و در نهایت توزیع شرطی با استفاده از معماری \lr{MAF} که یکی از شبکه‌های رایج و قوی در این خانواده است، یاد گرفته و ارزیابی شد.
\\
بر خلاف انتظارات، علی رقم عملکرد بهتر در مقایسه با مدل مشابه دارای فضای نهان، اما چندان موفق عمل نکرد. با توجه به آزمایشات انجام شده  این مشکل به دلیل کم بودن ظرفیت مدل مولد شرطی و آموزش به روش \maxlikelihood{}، پدیده \meanseeking{} رخ داده و چندان توانایی جدا کردن توزیع دو مقدار شرط از یکدیگر را ندارد. از سوی دیگر نیز این طور به نظر می‌رسد که تولید جملات با استفاده از فضای نهان چندان معقول نیست. چرا که بسیاری از مفاهیم باید در فضای نهان ذخیره شوند در حالی که این فضا محدود است و احتمالا توانایی مدل کردن تمام جزئیات را نخواهد داشت.